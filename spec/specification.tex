\documentclass[12pt]{article}
\usepackage[margin=1.5cm]{geometry}
\usepackage[T1]{fontenc}
\renewcommand{\contentsname}{INDEX}
\usepackage{longtable,multirow,fancyvrb}
\usepackage{hyperref}
\usepackage{mathtools}
\usepackage{fontspec}
\DeclarePairedDelimiter{\ceil}{\lceil}{\rceil}
\DeclarePairedDelimiter{\floor}{\lfloor}{\rfloor}
\usepackage[shortlabels]{enumitem}
\newcommand{\solution}{oindent \textbf{Solution: }}
\usepackage{amsfonts,amssymb,amsmath,float,graphicx,enumitem,titling}
\setlength{\droptitle}{-5em}   % This is your set screw
\title{Compilers-2\\Group-12}
\date{}
\begin{document}
% fix error in this document?

\begin{figure}[H]
    \centering 
    \includegraphics[scale=0.1]{./logo.png}
    \label{fig:my_label}
\end{figure}
\vspace{1cm}
\begin{center}
    {\fontsize{35}{30}\selectfont Compilers-II}\\
    \vspace{2cm}
    {\fontsize{35}{30}\selectfont CPlex}\\
    \vspace{0.5cm}
    {\fontsize{25}{30}\selectfont language-specification}\\
    \vspace{1cm}
    {\fontsize{30}{30}\selectfont By Group-12}\\
\end{center}
\pagebreak
\begin{center}
    \tableofcontents
\end{center}
\pagebreak
\section{Introduction}
\subsection{Motivation}
This language's inspiration stems from two distinct sources. For many years, compilers were responsible for converting high-level code into binary or assembly code. This put a program's effectiveness wholly dependent on the programmer's coding abilities. Compilers shouldn't be left behind as AI has recently advanced in all industries. More than only translators must be included in the compilers. Many studies are being conducted on the subject everywhere in the world. The more optimizations we do in the compailer, faster we get the output.

                         We took it as an opportunity and we have decided to build a new language called CPlex. CPlex is primarily used for Complex Numbers computations. Complex numbers are mainly used in Electrical Engineering, Signal Processing, Quantum Mechanics, Computer Graphics, Control systems etc. Many scientists make research on the areas mentioned above. This motivates us to build a programming language based on Complex Numbers.
\subsection{Goal}
CPlex aims to extend its capabilities to a broader, generalized domain. CPlex offers built-in support for complex numbers, facilitating arithmetic operations like addition, subtraction, multiplication, and division. 
\section{Data types}
Hello world 
\section{Operators and expressions}
Hello world 
\section{Lexical specifications}
Hello world 
\section{Declarations}
\subsection{Integers:}

\begin{enumerate}
    \item \textbf{int a, b, c;} \\
        Integers without initialization. a, b,c can be declared without initialization.
    \item  \textbf{int a = 5;}\\
        Integers with initialization. a can be declared with initializing.
\end{enumerate}

\subsection{Decimals:}
\begin{enumerate}
    \item \textbf{double a, b, c;} \\
        Decimal numbers without initialization. a, b,c can be declared without initialization.
    \item  \textbf{double a = 5.6;}\\
        Decimal numbers with initialization. a is assigned to 5.6 .
\end{enumerate}
\subsection{Complex declarations:}
\subsubsection{Complex Numbers without real part:}
\begin{enumerate}
    \item \textbf{cint a(3), b(4), c(10);}\\
Integer type complex numbers.
The above statements makes declarations as a = 3i, b = 4i, c = 10i.
    \item \textbf{cdouble c(3.4), d(4.7), e(10.03);} \\
The above statements makes declarations as c = 3.4i, d = 4.7i, e = 10.03i.
\end{enumerate}

\subsubsection{Complex Numbers with real part:}
\begin{enumerate}
    \item \textbf{cint a(3, 4);}\\
The above statement declares a complex number a = 3 + 4i.
    \item \textbf{cdouble a(3.5, 4.7);}\\
The above statement declares a complex number a = 3.5 + 4.7i.
\end{enumerate}
\subsection{Arrays:}
\subsubsection{Integer Arrays:} 
\begin{enumerate}
    \item \textbf{int a[10];}\\
Integer array without initialization.
The above statement allocates a with size of 10 where we can declare int datatype numbers.
\item \textbf{int a(23)[10];} \\
Integer array with initialization.
The above statement allocates a with size of 10 and are initialized to 23.
\end{enumerate}

\subsubsection{Double Arrays:} 
\begin{enumerate}
    \item \textbf{double d[25]; } \\
Double array without declaration.
The above statement allocates size of 25 where we can declare double datatype numbers.
\item \textbf{double d(3.8)[25]; } \\
Double array with declaration.
The above statement allocates a with size of 25 and are initialized to 3.8 .
\end{enumerate}
\subsubsection{Complex Number arrays:}
 \begin{enumerate}
     \item \textbf{cint a(3)[10];} \\
This statement declares an array of size 10 with each value assigned to 3.
     \item  \textbf{cint a(3, 4)[20];}\\ 
This statement declares an array of size 20 with each value assigned to a complex number 3 + 4i. If user wants to declare any complex number other 
than 3 + 4i, then he/she has to declare manually below the declaration part.
 \end{enumerate}
 \begin{enumerate}
    \item \textbf{cdouble a(3.4)[10];} \\
This statement declares an array of size of 10 with each value assigned to 3.4 .
\item \textbf{cdouble a(3.4, 6.93)[5];} \\
This statement declares an array of size of 5 with each value assigned to 3.4 + 6.93i.
 \end{enumerate}

\subsection{Function Declaration SYNTAX:}
    \begin{BVerbatim} 
FUNC_NAME RETURN_TYPE : (DTYPE VAR1, DTYPE VAR2)  {
   /*
   CODE TO BE WRITTEN HERE
   */
}
    \end{BVerbatim}

First we declare function name then we mention return type and then we have a Colon after that we declare arguments separated by comma. After the parenthesis we write as usual code. Parenthesis are completely optional for the arguement declaration.


\section{Statements}
Hello world 
\section{Built-in functions}
\subsection{Inbuilt complex functions:}
\begin{itemize}
    \item \texttt{ real double : cdouble c}: Returns the real part of the complex number \texttt{c}.
    \item \texttt{img double : (cdouble c)}: Returns the imaginary part of the complex number \texttt{c}.
    \item \texttt{ pow cdouble : (cdouble base,double exponent)}: Returns the complex number ${(base)}^{(exponent)}$. This is done by using De Moivre's formula.
    \item \texttt{ polar void :(cdouble c)}: Prints the polar form of a complex number \texttt{c}. Given a complex number $c=a+ib$ the polar form looks $c=r(e^{i \theta })$ (Where $\theta$ is the argument of the complex number and $r$ is the modulus of the complex number).
    \item \texttt{conjugate cdouble : (cdouble c)}: Returns the conjugate of the complex number \texttt{c}. Given a complex number $c=a+ib$ the conjugate looks like $c=a-ib$.
    \item  \texttt{mod double : (cdouble c)}: Returns the modulus of the complex number \texttt{c}. Given a complex number $c=a+ib$ the modulus looks like $c=\sqrt{a^2+b^2}$.
    \item \texttt{arg double :(cdouble c)}: Returns the argument of the complex number \texttt{c}. Given a complex number $c=a+ib$ the argument looks like $c=\tan^{-1}(\frac{b}{a})$.
    \item  \texttt{angle double : (cdouble c1,cdouble c2)}: Returns the angle between the complex numbers \texttt{c1} and \texttt{c2}. Given two complex numbers $c_1=a_1+b_1i$ and $c_2=a_2+b_2i$ the angle between them looks like $c=\tan^{-1}(\frac{b_2-b_1}{a_2-a_1})$.
    \item \texttt{dist double : (cdouble c1,cdouble c2)}: Returns the distance between the complex numbers \texttt{c1} and \texttt{c2}. Given two complex numbers $c_1=a_1+b_1i$ and $c_2=a_2+b_2i$ the distance between them looks like $c=\sqrt{(a_2-a_1)^2+(b_2-b_1)^2}$.
    \item \texttt{ cprint void: (cdouble c)}: Prints the complex number \texttt{c} in the form $a+ib$.
\end{itemize}
\subsection{Geometry related:}
\begin{itemize}
    \item \texttt{rotate cdouble : (cdouble c,cdouble origin,double angle)}: Returns the complex number \texttt{c} rotated by an angle \texttt{angle} about the point \texttt{origin}. The rotation is done in the counter-clockwise direction.    
    \item \texttt{dist double :(cdouble c1,cdouble c2)}: Returns the distance between the complex numbers \texttt{c1} and \texttt{c2}. Given two complex numbers $c_1=a_1+b_1i$ and $c_2=a_2+b_2i$ the distance between them looks like $c=\sqrt{(a_2-a_1)^2+(b_2-b_1)^2}$.
    \item \texttt{get\_line void :(cdouble c1,cdouble c2,double *a,double *b,double *c)}: Given two complex numbers $c_1=a_1+b_1i$ and $c_2=a_2+b_2i$ this function prints the line $ax+by+c=0$ passing through the points $c_1$ and $c_2$.
    \item \texttt{is\_traingle bin :(cdouble c1,cdouble c2,cdouble c3)}: Given three complex numbers $c_1=a_1+b_1i$,$c_2=a_2+b_2i$ and $c_3=a_3+b_3i$ this function returns true if the points $c_1$,$c_2$ and $c_3$ form a triangle else false.
    \item \texttt{ get\_centroid cdouble:(cdouble c1,cdouble c2,cdouble c3)}: Given three complex numbers $c_1=a_1+b_1i$,$c_2=a_2+b_2i$ and $c_3=a_3+b_3i$ this function returns the centroid of the triangle formed by(if exists) the points $c_1$,$c_2$ and $c_3$.
    \item \texttt{get\_circumcenter cdouble :(cdouble c1,cdouble c2,cdouble c3)}: Given three complex numbers $c_1=a_1+b_1i$,$c_2=a_2+b_2i$ and $c_3=a_3+b_3i$ this function returns the circumcenter of the triangle formed by(if exists) the points $c_1$,$c_2$ and $c_3$.
    \item \texttt{get\_orthocenter cdouble :(cdouble c1,cdouble c2,cdouble c3)}: Given three complex numbers $c_1=a_1+b_1i$,$c_2=a_2+b_2i$ and $c_3=a_3+b_3i$ this function returns the orthocenter of the triangle formed by(if exists) the points $c_1$,$c_2$ and $c_3$.
    \item \texttt{get\_incenter cdouble :(cdouble c1,cdouble c2,cdouble c3)}: Given three complex numbers $c_1=a_1+b_1i$,$c_2=a_2+b_2i$ and $c_3=a_3+b_3i$ this function returns the incenter of the triangle formed by(if exists) the points $c_1$,$c_2$ and $c_3$.
    \item \texttt{get\_excenter cdouble :(cdouble c1,cdouble c2,cdouble c3)}: Given three complex numbers $c_1=a_1+b_1i$,$c_2=a_2+b_2i$ and $c_3=a_3+b_3i$ this function returns the excenter of the triangle formed by(if exists) the points $c_1$,$c_2$ and $c_3$.
    \item \texttt{ get\_area double :(cdouble c1,cdouble c2,cdouble c3)}: Given three complex numbers $c_1=a_1+b_1i$,$c_2=a_2+b_2i$ and $c_3=a_3+b_3i$ this function returns the area of the triangle formed by(if exists) the points $c_1$,$c_2$ and $c_3$.
    \item \texttt{get\_perimeter double :(cdouble c1,cdouble c2,cdouble c3)}: Given three complex numbers $c_1=a_1+b_1i$,$c_2=a_2+b_2i$ and $c_3=a_3+b_3i$ this function returns the perimeter of the triangle formed by(if exists) the points $c_1$,$c_2$ and $c_3$.
\end{itemize}
\section{Example programs}
\subsection{Example program 1:}
\begin{figure}[H]
    \label{ex_program_1}
    \centering
    \begin{BVerbatim}
my_centroid cdouble : cdouble c1,cdouble c2,cdouble c3  {
    cdouble centroid;
    centroid = (c1+c2+c3)/3;
    return centroid;
}
main int :  {
    cint a(3,4);
    cint b(5,5),c(-101,100);
    cdouble centroid;
    centroid = my_centroid(a,b,c);
    choice(centroid eq get_centroid(a,b,c)) {
        cprint(centroid);
    } 
    default {
        cprint(is_triangle(a,b,c));
    }
    return 0;
}
    \end{BVerbatim}
    \caption{Output for table 'department' and k=10}
    \end{figure}
\subsection{Example program 2:}
\begin{figure}[H]
    \label{ex_program_2}
    \centering
    \begin{BVerbatim}
        main int :  {
            cint a(3,4);
            cint b(5,5),c(-101,100);
            cdouble centroid;
            centriod = get_centroid(a,b,c);
            cprint(centroid);
            circumcente= get_circumcenter(a,b,c);
            cprint(circumcenter);
            orthocenter = get_orthocenter(a,b,c);
            cprint(orthocenter);
            choice (dist(centriod,circumcenter) eq dist(orthocenter,centroid)*2){
                cprint(1); //ratio verified
            }
            default {
                cprint(-1);
            }
            //circum centriod orthocenter
            //      2        1
            return 0;
        }
            \end{BVerbatim}
    \caption{Output for table 'department' and k=10}
    \end{figure}
\end{document}