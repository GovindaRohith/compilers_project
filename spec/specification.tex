\documentclass[12pt]{article}
\usepackage[margin=1.5cm]{geometry}
\usepackage[T1]{fontenc}
\renewcommand{\contentsname}{INDEX}
\usepackage{longtable,multirow,fancyvrb}
\usepackage{hyperref}
\usepackage{mathtools}
\usepackage{fontspec}
\DeclarePairedDelimiter{\ceil}{\lceil}{\rceil}
\DeclarePairedDelimiter{\floor}{\lfloor}{\rfloor}
\usepackage[shortlabels]{enumitem}
\newcommand{\solution}{oindent \textbf{Solution: }}
\usepackage{amsfonts,amssymb,amsmath,float,graphicx,enumitem,titling}
\setlength{\droptitle}{-5em}   % This is your set screw
\title{Compilers-2\\Group-12}
\date{}
\begin{document}
% fix error in this document?

\begin{figure}[H]
    \centering 
    \includegraphics[scale=0.1]{./logo.png}
    \label{fig:my_label}
\end{figure}
\vspace{1cm}
\begin{center}
    {\fontsize{35}{30}\selectfont Compilers-II}\\
    \vspace{2cm}
    {\fontsize{35}{30}\selectfont CPlex}\\
    \vspace{0.5cm}
    {\fontsize{25}{30}\selectfont language-specification}\\
    \vspace{1cm}
    {\fontsize{30}{30}\selectfont By Group-12}\\
\end{center}
\pagebreak
\begin{center}
    \tableofcontents
\end{center}
\pagebreak
\section{Introduction}
\subsection{Motivation}
motivation here
\subsection{Goal}
goal here
\section{Data types}
Hello world 
\section{Operators and expressions}
Hello world 
\section{Lexical specifications}
Hello world 
\section{Declarations}
Hello world 
\section{Statements}
Hello world 
\section{Built-in functions}
\subsection{Inbuilt complex functions:}
\begin{itemize}
    \item \texttt{ real double : cdouble c}: Returns the real part of the complex number \texttt{c}.
    \item \texttt{img double : (cdouble c)}: Returns the imaginary part of the complex number \texttt{c}.
    \item \texttt{ pow cdouble : (cdouble base,double exponent)}: Returns the complex number ${(base)}^{(exponent)}$. This is done by using De Moivre's formula.
    \item \texttt{ polar void :(cdouble c)}: Prints the polar form of a complex number \texttt{c}. Given a complex number $c=a+ib$ the polar form looks $c=r(e^{i \theta })$ (Where $\theta$ is the argument of the complex number and $r$ is the modulus of the complex number).
    \item \texttt{conjugate cdouble : (cdouble c)}: Returns the conjugate of the complex number \texttt{c}. Given a complex number $c=a+ib$ the conjugate looks like $c=a-ib$.
    \item  \texttt{mod double : (cdouble c)}: Returns the modulus of the complex number \texttt{c}. Given a complex number $c=a+ib$ the modulus looks like $c=\sqrt{a^2+b^2}$.
    \item \texttt{arg double :(cdouble c)}: Returns the argument of the complex number \texttt{c}. Given a complex number $c=a+ib$ the argument looks like $c=\tan^{-1}(\frac{b}{a})$.
    \item  \texttt{angle double : (cdouble c1,cdouble c2)}: Returns the angle between the complex numbers \texttt{c1} and \texttt{c2}. Given two complex numbers $c_1=a_1+b_1i$ and $c_2=a_2+b_2i$ the angle between them looks like $c=\tan^{-1}(\frac{b_2-b_1}{a_2-a_1})$.
    \item \texttt{dist double : (cdouble c1,cdouble c2)}: Returns the distance between the complex numbers \texttt{c1} and \texttt{c2}. Given two complex numbers $c_1=a_1+b_1i$ and $c_2=a_2+b_2i$ the distance between them looks like $c=\sqrt{(a_2-a_1)^2+(b_2-b_1)^2}$.
    \item \texttt{ cprint void: (cdouble c)}: Prints the complex number \texttt{c} in the form $a+ib$.
\end{itemize}
\subsection{Geometry related:}
\begin{itemize}
    \item \texttt{rotate cdouble : (cdouble c,cdouble origin,double angle)}: Returns the complex number \texttt{c} rotated by an angle \texttt{angle} about the point \texttt{origin}. The rotation is done in the counter-clockwise direction.    
    \item \texttt{dist double :(cdouble c1,cdouble c2)}: Returns the distance between the complex numbers \texttt{c1} and \texttt{c2}. Given two complex numbers $c_1=a_1+b_1i$ and $c_2=a_2+b_2i$ the distance between them looks like $c=\sqrt{(a_2-a_1)^2+(b_2-b_1)^2}$.
    \item \texttt{get\_line void :(cdouble c1,cdouble c2,double *a,double *b,double *c)}: Given two complex numbers $c_1=a_1+b_1i$ and $c_2=a_2+b_2i$ this function prints the line $ax+by+c=0$ passing through the points $c_1$ and $c_2$.
    \item \texttt{is\_traingle bin :(cdouble c1,cdouble c2,cdouble c3)}: Given three complex numbers $c_1=a_1+b_1i$,$c_2=a_2+b_2i$ and $c_3=a_3+b_3i$ this function returns true if the points $c_1$,$c_2$ and $c_3$ form a triangle else false.
    \item \texttt{ get\_centroid cdouble:(cdouble c1,cdouble c2,cdouble c3)}: Given three complex numbers $c_1=a_1+b_1i$,$c_2=a_2+b_2i$ and $c_3=a_3+b_3i$ this function returns the centroid of the triangle formed by(if exists) the points $c_1$,$c_2$ and $c_3$.
    \item \texttt{get\_circumcenter cdouble :(cdouble c1,cdouble c2,cdouble c3)}: Given three complex numbers $c_1=a_1+b_1i$,$c_2=a_2+b_2i$ and $c_3=a_3+b_3i$ this function returns the circumcenter of the triangle formed by(if exists) the points $c_1$,$c_2$ and $c_3$.
    \item \texttt{get\_orthocenter cdouble :(cdouble c1,cdouble c2,cdouble c3)}: Given three complex numbers $c_1=a_1+b_1i$,$c_2=a_2+b_2i$ and $c_3=a_3+b_3i$ this function returns the orthocenter of the triangle formed by(if exists) the points $c_1$,$c_2$ and $c_3$.
    \item \texttt{get\_incenter cdouble :(cdouble c1,cdouble c2,cdouble c3)}: Given three complex numbers $c_1=a_1+b_1i$,$c_2=a_2+b_2i$ and $c_3=a_3+b_3i$ this function returns the incenter of the triangle formed by(if exists) the points $c_1$,$c_2$ and $c_3$.
    \item \texttt{get\_excenter cdouble :(cdouble c1,cdouble c2,cdouble c3)}: Given three complex numbers $c_1=a_1+b_1i$,$c_2=a_2+b_2i$ and $c_3=a_3+b_3i$ this function returns the excenter of the triangle formed by(if exists) the points $c_1$,$c_2$ and $c_3$.
    \item \texttt{ get\_area double :(cdouble c1,cdouble c2,cdouble c3)}: Given three complex numbers $c_1=a_1+b_1i$,$c_2=a_2+b_2i$ and $c_3=a_3+b_3i$ this function returns the area of the triangle formed by(if exists) the points $c_1$,$c_2$ and $c_3$.
    \item \texttt{get\_perimeter double :(cdouble c1,cdouble c2,cdouble c3)}: Given three complex numbers $c_1=a_1+b_1i$,$c_2=a_2+b_2i$ and $c_3=a_3+b_3i$ this function returns the perimeter of the triangle formed by(if exists) the points $c_1$,$c_2$ and $c_3$.
\end{itemize}
\section{Example programs}
\subsection{Example program 1:}
\begin{figure}[H]
    \label{ex_program_1}
    \centering
    \begin{BVerbatim}
my_centroid cdouble : cdouble c1,cdouble c2,cdouble c3  {
    cdouble centroid;
    centroid = (c1+c2+c3)/3;
    return centroid;
}
main int :  {
    cint a(3,4);
    cint b(5,5),c(-101,100);
    cdouble centroid;
    centroid = my_centroid(a,b,c);
    choice(centroid eq get_centroid(a,b,c)) {
        cprint(centroid);
    } 
    default {
        cprint(is_triangle(a,b,c));
    }
    return 0;
}
    \end{BVerbatim}
    \caption{Output for table 'department' and k=10}
    \end{figure}
\subsection{Example program 2:}
\begin{figure}[H]
    \label{ex_program_2}
    \centering
    \begin{BVerbatim}
        main int :  {
            cint a(3,4);
            cint b(5,5),c(-101,100);
            cdouble centroid;
            centriod = get_centroid(a,b,c);
            cprint(centroid);
            circumcente= get_circumcenter(a,b,c);
            cprint(circumcenter);
            orthocenter = get_orthocenter(a,b,c);
            cprint(orthocenter);
            choice (dist(centriod,circumcenter) eq dist(orthocenter,centroid)*2){
                cprint(1); //ratio verified
            }
            default {
                cprint(-1);
            }
            //circum centriod orthocenter
            //      2        1
            return 0;
        }
            \end{BVerbatim}
    \caption{Output for table 'department' and k=10}
    \end{figure}
\end{document}